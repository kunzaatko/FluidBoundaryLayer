\fontsize{10pt}{15pt}\selectfont
\setcounter{example}{27}
\begin{example}
	Řešte metodou střelby Blasiovu rovnici mezní vrstvy
	\begin{gather*}
		y''' + y y'' + \lambda \left( 1 - y'^{2} \right)  = 0,
		\qq{kde } x \in (0,10) \qcomma \lambda \in \langle 0, 0.5 \rangle \\
		y(0) = y'(0) = 0 \qcomma y'(10) = 1
	\end{gather*}
	\smallskip
	\hrule
	\medskip
	Nejprve musíme diferenciální rovnici třetího řádu rozložit na tři obyčejné diferenciální rovnice prvního řádu. To provedeme substitucemi:
	\begin{equation}
		\label{substituce}
		y = z^{(0)}(x) \qcomma
		y' = z^{(1)}(x) = \dv{z^{(0)}}{x} \qcomma
		y'' = z^{(2)}(x) = \dv{z^{(1)}}{x} \qcomma
		y''' = z^{(3)}(x) = \dv{z^{(2)}}{x}.
	\end{equation}
	Dosazením \ref{substituce} do původní rovnice pak získáváme soustavu diferenciálních rovnic prvního řádu pro funkce $z^{(0)}, z^{(1)}, z^{(2)}$:
	\begin{equation}
		\begin{aligned}
			z^{(0)}(x) & = y                                                                                   \\
			z^{(1)}(x) & = \dv{z^{(0)}}{x}                                                                     \\
			z^{(2)}(x) & = \dv{z^{(1)}}{x}                                                                     \\
			z^{(3)}(x) & = \dv{z^{(2)}}{x}(x) & = -\lambda \left( 1 - {z^{(1)}}^{2} \right)  - z^{(2)}z^{(0)}.
		\end{aligned}
	\end{equation}
	Počáteční podmínky se převedou na $z^{(0)}(0) = 0$ a $z^{(1)}(0) = 0$ a koncová podmínka na $z^{(1)}(10) = 1$.

	\medskip

	Tuto soustavu budeme vnímat jako vektorovou funkci od $x$, a definujeme $\vec{f}(x)$ jako její $\grad_x$ konkrétně
	\begin{equation}
		\vec{Z}(x) =
		\begin{pmatrix}
			z^{(0)}(x) \\
			z^{(1)}(x) \\
			z^{(2)}(x) \\
		\end{pmatrix} \qcomma
		\vec{f}\left(x,\vec{Z}\right) =
		\begin{pmatrix}
			\dv{z^{(0)}}{x}\eval_{x,\vec{Z}} \\
			\dv{z^{(1)}}{x}\eval_{x,\vec{Z}} \\
			\dv{z^{(2)}}{x}\eval_{x,\vec{Z}} \\
		\end{pmatrix}  =
		\begin{pmatrix}
			z^{(1)}(x,\vec{Z}) \\
			z^{(2)}(x,\vec{Z}) \\
			z^{(3)}(x,\vec{Z}) \\
		\end{pmatrix} =
		\begin{pmatrix}
			z^{(1)} \\
			z^{(2)} \\
			-\lambda \left( 1 - {z^{(1)}}^{2} \right)  - z^{(2)}z^{(0)}
		\end{pmatrix}.
	\end{equation}
	Numerické řešení budeme provádět jednosměrnou metodou střelby z bodu $x_{0} = \vec{0}$ a řešení diferenciálních rovnic s pokusnými počátečními podmínkami Rundge-Kutta metodou čtvrtého řádu tedy:
	\begin{equation}
		\begin{aligned}
			\vec{k}_{1} & = h \cdot \vec{f}\left(  x_{n}, \vec{Z}(x_{n}) \right)                                       \\
			\vec{k}_{2} & = h \cdot \vec{f}\left(  x_{n} + \frac{h}{2}, \vec{Z}(x_{n}) + \frac{\vec{k}_{1}}{2} \right) \\
			\vec{k}_{3} & = h \cdot \vec{f}\left( x_{n} + \frac{h}{2}, \vec{Z}(x_{n}) + \frac{\vec{k}_{2}}{2}  \right) \\
			\vec{k}_{4} & = h \cdot \vec{f}\left( x_{n} + h, \vec{Z}(x_{n}) + \vec{k}_{3}  \right)
		\end{aligned}
	\end{equation}
	kde finální tvar odhadu $\vec{Z}(x_{n+1})$ je
	\begin{equation}
		\vec{Z}(x_{n+1}) = \vec{Z}(x_{n}) + \frac{1}{6}  \left(\vec{k}_{1} +  2\vec{k}_{2} + 2\vec{k}_{3} + \vec{k}_{4} \right).
	\end{equation}
	Pro dopředný chod potřebujeme tři parametry v počátečním bodě a ty vezmeme jako
	\begin{equation}
		\vec{Z}(0) =
		\begin{pmatrix}
			z^{(0)}(0) \\
			z^{(1)}(0) \\
			z^{(2)}(0)
		\end{pmatrix},  \qq{přičemž} z^{(0)}(0) = z^{(1)}(0) = 0,
	\end{equation}
	tedy s našimi počátečními podmínkami hledáme v prostoru počátečních parametrů
	\begin{equation}
		\vec{Z}(0) \in \left(
		\begin{pmatrix}
			0 \\
			0 \\
			1 \\
		\end{pmatrix}
		\right)_{\lambda}.
	\end{equation}
	Koncový bod potom musí splňovat podmínku $f'(10) = \dv{z^{(0)}}{x}\eval_{x=10} = z^{(1)}(10) = 1$, tedy koncové řešení je z prostoru
	\begin{equation}
		\vec{Z}(10) \in
		\begin{pmatrix}
			0 \\
			1 \\
			0 \\
		\end{pmatrix}  +
		\left(
		\begin{pmatrix}
			1 \\
			0 \\
			0 \\
		\end{pmatrix},
		\begin{pmatrix}
			0 \\
			0 \\
			1 \\
		\end{pmatrix}
		\right)_{\lambda}.
	\end{equation}
	Samotná střela potom probíhá tak, že začneme se dvěma zadanými počátečními podmínkami a jedním zvoleným algoritmem hledání kořene (viz. dále) v bodě $x = 0$ a vektor $\vec{Z}(0)$ zvoleným počtem kroků s pomocí Rundge-Kuttova předpisu pro změnu $\vec{Z}$ posouváme až do $\vec{Z}(10)$.
	\medskip
	K minimalizaci chyby splnění koncové podmínky, které je ekvivalentní hledání kořene funkce následující funkce $B_{2}$ od $\vec{Z}_{0}$, kde ale složky počáteční podmínky $z^{(0)}(0) = z^{(1)}(0) = 0$ ze zadání, tedy funkce bude pouze od $z^{(2)(0)}$.
	\begin{equation}
		B_{2}(x, \vec{Z}_{0}) \eval_{x = 10}= B_{2}(x, z^{(2)}_{0}) \eval_{x=0}= z^{(1)}(x) - 1 \eval_{x=10}= 0.
	\end{equation}
	Zkusíme na to použít naivní Newtonovou-Raphsonovou metodou. Přičemž pro vyhodnocení $z^{(1)}(10) - 1$ musíme vždy najít hodnotu $z^{(1)}(10)$ střelbou s proměnným počátečním parametrem $z^{(2)}_{0}$.

	\smallskip

	Protože máme prostor počátečních parametrů pouze jednodimenzionální, budeme k tomu potřebovat pouze parciální derivaci podle $z_{0}^{(2)}$:
	\begin{equation}
		\pdv{B_{2}}{z^{(2)}_{0}} \approx \frac{B_{2}(z^{(2)}_{0} + \Delta z^{(2)}_{0}) - B_{2}(z^{(2)}_{0})}{\Delta z^{(2)}_{0}}.
	\end{equation}
	Ta se získá dvěma střelami s počátečními parametry $z^{(2)}_{0} + \Delta z^{(2)}_{0}$ a $z^{(2)}_{0}$ pro vyhodnocení $z^{(1)}(x)\eval_{x=10} \eqqcolon z^{(1)}_{10}$ a následným vyhodnocením chybové funkce $B_{2}$.

	\bigskip

	Implementací se ukázalo, že naivní Newtonova-Raphsonova metoda na hledání kořene není vhodná, jelikož není v tomto příkladu zjevně stabilní. Při splnění koncové podmínky se totiž blíží parciální derivace $B_{2}$ podle $z^{(2)}_{0}$ nule neboli $\pdv{z^{(1)}_{10}}{z^{(2)}_{0}}$ = 0. Řešení je tedy v bodě extrému $B_{2}$ nebo v inflexním bodě s nulovou derivací. Tam naivní Newtonova-Raphsonova metoda na hledání kořene fungovat nebude, protože bude dělat blízko u kořene velké skoky.
	\begin{remark}
		Ekvivalentní příklad by bylo pro případ extrému hledání kořene paraboly $y(x) = ax^2\qcomma a \neq 0$ a pro případ inflexního bodu s nulovou derivací hledání kořene kubického monomu $y(x) = ax^3\qcomma a \neq 0$, kde není zaručeno, že interval omezující přítomnost kořene se zachová v jednotlivých iteracích.
	\end{remark}

	Tyto dva případy se ještě liší v tom, jakou metodu budeme moci pro hledání kořene využít. Pokud se jedná o případ, že je v bodě $z^{(2)}_{0}$ pouze inflexní bod, budeme schopni použít například metodu půlení intervalu, která je v tomto případě relativně rychlá protože vyžaduje malý počet numericky náročných vyhodnocení střel. Jelikož jistě z tvaru $f(x, \vec{Z})$ platí $\lim_{z^{(2)}_{0} \to \infty} B_{2}(10) < 0$ a z testování implementace vyšlo, že existují i počáteční hodnoty s kladnou hodnotou chybové funkce $B_{2}$ (například, $z^{(2)}_{0} = 0$), je dobrá šance, že tato metoda bude na příklad fungovat.

	\begin{remark}
		Mohli bychom použít metodu sečen, která konverguje rychleji, ale pro využití metody sečen by bylo nutné ve vhodnou chvíli přepnout na jinou metodu, jelikož pokud bychom chtěli přesný výsledek, bude nám na malých intervalech vznikat numerická chyba díky dělení rozdílem funkčních hodnot na krajích intervalu.

		(Nejsem si jistý, jestli toto by byl problém, vzhledem k tomu, že dělíme jistě menší hodnotu. Je možné, že numerická chyba převáží a výsledek bude větší než 1 tedy bod ležet mimo interval?)

		(Díky této numerické chybě je opět porušená záruka, že interval z další iterace je podmnožinou intervalu z předchozí iterace? Musela by se implementovat mezipaměť mezi jednotlivými iteracemi a logika, pro porušení této podmínky.)
	\end{remark}

	Co se týká implementace půlení intervalu, tak pro zvolený interval $(a,b)$, který obsahuje kořen vždy vyhodnotíme funkci (střelu) v $ c \coloneqq \frac{(a+b)}{2} = z^{(2)}_{0}$ a poté jeden z bodů $a,b$ nahradíme $c$, tak aby ve výsledném intervalu zůstal kořen (aby vyčíslení funkce na hranicích intervalu měla opačná znaménka).\pagebreak
\end{example}
