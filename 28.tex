\documentclass{article}
\usepackage{amsmath,amssymb,amsthm}
\usepackage{physics}
\usepackage[left=1cm,right=1cm,top=1cm,bottom=2cm]{geometry}
\newtheorem{example}{Příklad}
\begin{document}
\setcounter{example}{27}
\begin{example}
    Řešte metodou střelby Blasiovu rovnici mezní vrstvy
    \begin{gather*}
        y''' + y y'' + \lambda \left( 1 - y'^{2} \right)  = 0,
        \qq{kde } x \in (0,10) \qcomma \lambda \in \langle 0, 0.5 \rangle \\
        y(0) = y'(0) = 0 \qcomma y'(10) = 1
    \end{gather*}
    \smallskip
    \hrule
    \medskip
    Nejprve musíme diferenciální rovnici třetího řádu rozložit na tři obyčejné diferenciální rovnice prvního řádu. To provedeme substitucemi:
    \begin{equation}
        \label{substituce}
        y = z^{(0)}(x) \qcomma
        y' = z^{(1)}(x) = \dv{z^{(0)}}{x} \qcomma
        y'' = z^{(2)}(x) = \dv{z^{(1)}}{x} \qcomma
        y''' = z^{(3)}(x) = \dv{z^{(2)}}{x}
    \end{equation}
    Dosazením \ref{substituce} do původní rovnice pak získáváme soustavu differenciálních rovnic prvního řádu:
    \begin{equation}
        \begin{aligned}
           z^{(0)}(x) &= y \\
           z^{(1)}(x) &= \dv{z^{(0)}}{x} \\
           z^{(2)}(x) &= \dv{z^{(1)}}{x} \\
           z^{(3)}(x) &= \dv{z^{(2)}}{x}(x) &= -\lambda \left( 1 - {z^{(1)}}^{2} \right)  - z^{(2)}z^{(0)}
        \end{aligned}
    \end{equation}
    Počáteční podmínky se převedou na $z^{(0)}(0) = 0$ a $z^{(1)}(0) = 0$ a koncová podmínka na $z^{(1)}(10) = 1$.
    
    Tuto soustavu budeme vnímat jako vektorovou funkci od $x$, a definujeme $\vec{f}(x)$ jako její derivace konkrétně 
    \begin{equation}
        \vec{Z}(x) = 
            \begin{pmatrix}
                z^{(0)}(x) \\ 
                z^{(1)}(x) \\
                z^{(2)}(x) \\
            \end{pmatrix} \qcomma 
        \vec{f}\left(x,\vec{Z}\right) = 
            \begin{pmatrix}
                \dv{z^{(0)}}{x}\eval_{x,\vec{Z}}\\
                \dv{z^{(1)}}{x}\eval_{x,\vec{Z}}\\
                \dv{z^{(2)}}{x}\eval_{x,\vec{Z}}\\
            \end{pmatrix}  =
            \begin{pmatrix} 
                z^{(1)}(x,\vec{Z}) \\
                z^{(2)}(x,\vec{Z}) \\
                z^{(3)}(x,\vec{Z}) \\
            \end{pmatrix} = 
            \begin{pmatrix} 
                z^{(1)} \\
                z^{(2)} \\
                -\lambda \left( 1 - {z^{(1)}}^{2} \right)  - z^{(2)}z^{(0)}
            \end{pmatrix} 
    \end{equation}
    Numerické řešení budeme provádět jednosměrnou metodou střelby z bodu $x_{0} = \vec{0}$ a řešení diferenciálních rovnic s pokusnými parametry Rundge-Kutta metodou čtvrtého řádu tedy:
    \begin{equation}
        \begin{aligned}
            \vec{k}_{1} &= h \cdot \vec{f}\left(  x_{n}, \vec{Z}(x_{n}) \right) \\ 
            \vec{k}_{2} &= h \cdot \vec{f}\left(  x_{n} + \frac{h}{2}, \vec{Z}(x_{n}) + \frac{\vec{k}_{1}}{2} \right) \\
            \vec{k}_{3} &= h \cdot \vec{f}\left( x_{n} + \frac{h}{2}, \vec{Z}(x_{n}) + \frac{\vec{k}_{2}}{2}  \right) \\
            \vec{k}_{4} &= h \cdot \vec{f}\left( x_{n} + h, \vec{Z}(x_{n}) + \vec{k}_{3}  \right)
        \end{aligned}
    \end{equation}
    kde finální tvar odhadu $\vec{Z}(x_{n+1})$ je
    \begin{equation}
\vec{Z}(x_{n+1}) = \vec{Z}(x_{n}) + \frac{1}{6}  \left(\vec{k}_{1} +  2\vec{k}_{2} + 2\vec{k}_{3} + \vec{k}_{4} \right)
    \end{equation}
    Pro dopředný chod potřebujeme čtyři parametry v počátečním bodě a ty vezmeme jako
    \begin{equation}
        \vec{Z}(0) = 
        \begin{pmatrix}
            z^{(0)}(0)\\
            z^{(1)}(0) \\
            z^{(2)}(0)
        \end{pmatrix} 
    \end{equation}
    tedy s našimi počátečními podmínkami hledáme v parametrickém prostoru
    \begin{equation}
        \vec{Z}(0) \in \left( 
            \begin{pmatrix}
                0 \\
                0 \\
                1 \\
            \end{pmatrix}
            \right)_{\lambda}
        \end{equation}
        Koncový bod potom musí splňovat podmínku 
        \begin{equation}
            \vec{Z}(10) \in 
            \begin{pmatrix}
                0 \\
                1 \\
                0 \\
            \end{pmatrix}  + 
            \left( 
                \begin{pmatrix} 
                    1 \\
                    0 \\
                    0 \\
                \end{pmatrix},
                \begin{pmatrix} 
                    0 \\
                    0 \\
                    1 \\
                \end{pmatrix}
            \right)_{\lambda}
        \end{equation}
    \end{example}
\end{document}
